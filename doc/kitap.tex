\documentclass[10pt,a4paper]{book}
\usepackage[turkish]{babel}
\usepackage[utf8]{inputenc}

\usepackage{makeidx}
\usepackage{graphicx}
\makeindex
\author{Ilker Manap}
\title{Flask Taslak Kullanımı}

\begin{document}
\maketitle
\tableofcontents
\chapter{Giriş}

Flask ile yazılmış uygulamaların bir web sunucu üzerinden sunulması çoğu kişi 
için başa çıkılamaz karmaşık işlemler gerektiren bir süreç gibidir. Birden 
fazla konuda doğru ayarlamalar yapmak gerektiği için, konuyu az bilenler 
tarafından yapılmaya çalışıldığında sorun çıkması ihtimali de yüksektir.


Bu belge ile, karmaşık görünen işlemlerin daha kolay yapılabilmesini sağla-maya 
çalışacağız. Düzgün ayarlanması gereken birden fazla nokta olduğundan, ayar gerektiren her bir bölüm için olabildiğince detaylı anlatılacaktır.





\chapter{Sanal Domain Nasıl Ayarlanır?}
DNS\index{DNS} ve nginx\index{nginx} ayarları
\section{DNS}
Bir web sunucusunun bir domain adına bağlı olarak çalışabilmesi için önce domain
alınan yerde ayarlar gerekir. Çoğunlukla dns servisi domain alınan yer üzerinden 
kullanılır. Yani domain satın aldığınız firma, aldığınız domaini yönetmek için size bir arayüz sağlar. Örnek olarak, godaddy firmasından deneme.com adresini aldığımızı düşünelim. Godaddy.com adresindeki bir yönetim paneli ile deneme.com için www.deneme.com, mail.deneme.com gibi yeni adresler ekleyebiliriz. Burada yazdığımız uygulamanın  hello.deneme.com adresinden sunulacağını varsayalım. 

Öncelikle,  hello.deneme.com adı için DNS panelinde A kaydı oluşturmalıyız. A kaydı ile, verdiğimiz adın sahip olacağı IP adresini tanımlamış oluruz. hello.de-neme.com için 56.35.3.122 gibi. Buradaki isim ve IP adresleri tamamen uydurmadır. 


Bir sanal sunucu aldığınızda, sanal sunucunuzun IP adresini bu amaçla kullanabilirsiniz. Bir sanal sunucuda çok sayıda farklı domain için web siteleri barındırmak mümkündür. Yani, hello.deneme.com için kullandığınız sunucuyu,  www.baskabirdomain.com için de sorun olmadan kullanabilirsiniz. İki web sitesinin birbirine karışmadan sunulması işini ise nginx ve apache gibi sunucular üstlenirler. Detayı nginx sunucusu için aşağıda anlatılacaktır.

DNS ile ilgili işlem tamamlandığında (web adresi için A kaydı tanımlaması), internete bağlı herhangi bir yerden, hello.deneme.com adresinini isim çözümlemesi yapılabiliyor olmalıdır. 


Dns detayları

\section{Nginx Ayarları}
Nginx

\chapter{Flask Kurulumu}



\chapter{Flask Uygulamanın Sunucuya Kurulması}

\printindex
\end{document}